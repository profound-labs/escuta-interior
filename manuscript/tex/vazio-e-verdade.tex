\chapter{\emph{Nada}, Vazio e Verdade}

A maioria dos praticantes do Budismo, independentemente da tradição,
estão familiarizados como que se chama `as três características da
existência' -- \emph{anicca, dukkha, anattā} (impermanência,
insatisfação e não-eu). Estas são as qualidades universais de todas as
experiências que surgem e desaparecem, e admitir a sua presença é o
aspecto mais vivo da meditação \emph{vipassanā.}

Existem, contudo, outras características universais da existência que, à
semelhança, podem ser usadas para ajudar o coração a libertar-se de toda
a limitação, peso e tensão. Duas destas características, operando como
num par, são chamadas de \emph{suññatā} e \emph{tathatā}, significando
respectivamente, vazio e Verdade. O termo `vazio' deriva de dizer `não'
ao fenómeno do mundo: «Não vou acreditar nisto -- é oco, vazio, não tem
nada, não é propriamente real.»

A `Verdade' é uma qualidade que corresponde a `vazio', da mesma forma
que a direita combina com a esquerda. Contrastando, porém, com a sua
parceira a sua natureza deriva de dizer `sim' ao universo. Poderá não
haver aqui nada de sólido, separado ou individual -- quer seja um
pensamento, um narciso ou uma montanha -- existe, contudo, algo, há aqui
uma Realidade Última que sustenta, permeia, abrange e constitui tudo. A
palavra `Verdade' expressa, assim, um apreço à natureza dessa Realidade,
e a sua tomada de consciência pode ser caracterizada pelo conhecimento e
materialização da presença do Incondicionado, o Imortal, ou
\emph{amata-dhamma}.

Quando se fala de vazio no \emph{Canon Pali,} as escrituras do mundo do
Budismo do sul, geralmente significa `o vazio do eu e do que pertence ao
eu', mas também se refere à insubstancialidade dos objectos. Quando se
desenvolve a capacidade da escuta interior e de seguir o som
\emph{nada}, potencia-se a compreensão quer do sujeito, quer do objecto,
quer do `eu', quer do `outro'.

Quando se firma na escuta do som \emph{nada} de uma forma estável, sendo
o seu tom cintilante de prata já uma presença constante, percebe-se quão
fácil é o reconhecimento da ausência de substância de todo o `eu',
`mim', ou 'meu', origem das atitudes e dos pensamentos, como já descrito
anteriormente. É como que uma luz brilhante, através da qual conseguimos
ver com clareza o oco nas bolhas de ar que flutuam.

De forma análoga, para todos os objectos mentais que são vividos -- tais
como o que vemos, ouvimos, cheiramos, saboreamos e tocamos, bem como as
memórias, planos, humores e ideias que surgem na mente -- a presença do
som \emph{nada} ajuda a iluminar a transparência de todos estes padrões
de consciência. O Buda disse-o desta forma:

A forma material é como espuma

Tocando uma bolha de água;

A percepção é só uma miragem,

Volições semelhantes a um tronco de planta,

Consciência, um truque de magia --

Assim diz o Parente do Sol.

Contudo, deve-se ponderar

Ou indagar cuidadosamente,

Afinal tudo é vazio e vago

Quando visto verdadeiramente

\emph{\textasciitilde{}S 22.95}

O som \emph{nada} também pode ajudar a relembrar a Verdade de todas as
experiências. Embora estes predicados possam parecer contraditórios,
será mais correcto dizer que são complementares. Quando se escuta
atentamente o som do silêncio e se permite que preencha o espaço
interior da consciência, a sua qualidade energética juntamente com a
riqueza informe de sua presença, é uma forte lembrança intuitiva da
condição da Verdade. É como se (pelo menos para quem fala inglês) o som
interior se expressasse num `\emph{iiiissssss}'\ldots{}, ou
\emph{`thussss\ldots{}}' infinito, que o traz de novo à realidade.

A Verdade é, por definição, conceptualmente difícil de explicar, possui
uma característica intrinsecamente incompreensível que pode parecer vaga
ou irreal, mas que, ironicamente, se torna parte necessária do seu
significado. É bem significativo o facto de Buda ter atribuído a si
próprio a expressão \emph{Tathāgata} -- que significa tanto ` O que
alcançou a `Verdade' como ` O que foi para a Verdade', dependendo da
interpretação. Assim, mesmo que a palavra `Verdade' possa trazer um tom
intangível (ao som do silêncio), tal é deliberado, e deve ser
reconhecida como expressão de uma realidade fundamental.

Pode-se fazer uma comparação com o mundo da matemática, usando o
conceito da raiz quadrada de -1. No mundo dos números reais não há
nenhum número inteiro que se possa multiplicar por si próprio para
produzir -1. Se, contudo, tal número existisse, todo o tipo de
possibilidades interessantes se abririam, como foi descoberto há muito
tempo, e desenvolvido pelos matemáticos do séc.XVIII.

É intrigante, mesmo sabendo que este número não existe no mundo real, só
tendo um estatuto imaginário, como se torna essencial na construção dos
deslocamentos de fase (atraso de propagação) dos osciladores, usado na
engenharia do som, estendendo-se o seu uso aos gráficos informatizados,
à robótica, ao processamento de sinal, simulações informatizadas e à
mecânica orbital.

Em conclusão, tal como acontece com a essência, mesmo que seja
indescritível, tem uma presença clara e demonstrável no mundo real. (1)

