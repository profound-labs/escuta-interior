\chapter{\emph{Nada} Permite Escutar Os Pensamentos}

Consoante vai desenvolvendo a escuta interior, usando a meditação
formal, começa a notar como o facto de escutar um objecto audível ajuda
a escutar de forma objectiva os pensamentos, os estados de espírito.

De certa forma, o tagarelar do pensamento não tem mais significado que o
sibilar cintilante do som \emph{nada}. São tão-somente as vibrações da
mente pensante dando forma a padrões conceptuais, só isso -- apenas uma
murmurante corrente de vibrações prolongada e contínua. Podemos, assim,
aprender a ouvir o nosso pensamento, tal como escutaríamos uma corrente
de água, uma queda de água, ou uma música coral de um bando de aves, com
o mesmo tipo de ausência de envolvimento ou de identificação. Não é mais
que um ribeiro murmurante na mente. Nada de mais - não suscita nada, nem
perturba.

Tudo isto é muito fácil de dizer, mas efectivamente temos a tendência
para nos envolver nas nossas histórias, não é? Adoramos histórias,
particularmente histórias sobre nós: o bem e o mal que fizemos, o
memorável, o doloroso, o lamentável, o que queremos fazer, o que
esperamos, o que receamos, o que os outros pensam de nós\ldots{}

Estes prezados padrões são todos manifestações do elemento 'eu', hábitos
de pensamento de toda uma vida, em termos de `eu', `mim 'e `meu'. Em
Pali são chamados de \emph{ahamkāra}, `criador de eu' e
\emph{mamankāra}, 'criador de meu', sendo eles as características-chave
da visão pessoal. Estes hábitos são o que mais repetida e efectivamente
dirige a nossa atenção para o domínio conceptual, levando, por sua vez,
à dispersão da mente. Se a história tiver um eu, torna-se muito mais
interessante que qualquer outra história remota. Tudo isto é
extremamente natural, um hábito básico de todos nós.

Por conseguinte, grande parte da meditação intuitiva, o desenvolvimento
de \emph{vipassanā,} só trata disso - aprender a reconhecer os hábitos
dos `criadores do eu e do meu' nos pensamentos que temos.
\emph{Ahamkāra} significa literalmente `feito de si próprio', enquanto
\emph{mamankāra} significa `feito de mim próprio'; a verdadeira intuição
é o reconhecimento desses hábitos, não se deixar levar pela história --
saber ver neles o vazio e a transparência, e libertar-se deles.

