\chapter{Os Atributos De \emph{Nada}}

Vários atributos do som \emph{nada} manifestam qualidades espirituais
muito úteis, algumas das quais permitem que sejam, pelo menos,
universalmente acessíveis e profícuas no que se refere, quanto mais não
seja, à concentração na respiração.

Primeiro, ao usar o som \emph{nada} como um objecto de meditação,
estimula-se a atitude de escuta e receptividade. Exige que, mais do que
dirigir uma actividade, se vivencie de coração aberto.

Segundo, o som não está sujeito ao controlo pessoal. Diferente da
respiração, que podemos alongar, encurtar, ou mudar segundo a nossa
vontade, não podemos tornar o som interior mais estridente ou mais
suave, fazer com que acabe ou comece, ou qualquer outra coisa. Podemos
focar-nos nele ou não, mas não está sujeito a direcções ou escolhas
pessoais. Na verdade, incita-nos naturalmente à realização na mais
profunda impessoalidade -- sem qualquer característica particular que
nos leve a pensar em ``mim'' ou ``meu''. Não é feminino, nem masculino,
novo nem velho, inteligente nem estúpido\ldots{}não tem tamanho nem
nacionalidade, nem cor nem língua\ldots{} existe, tão simplesmente, na
imparcialidade da Natureza.

Por fim, é energizante, possui uma qualidade que estimula naturalmente.
Quanto mais atenções lhe dermos, mais lúcida se torna a mente. Funciona
como um circuito de feedback positivo, de tal forma que, quanto maior
atenção se lhe der, mais reforçada será a capacidade de ficar atento.
Suporta, assim, o próprio acto de meditar, ao ajudar a mente a ficar
mais alerta.

