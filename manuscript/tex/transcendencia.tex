\chapter{\emph{Nada} Como Um Símbolo De Transcendência}

O som do silêncio é um objecto do domínio sensorial que reflecte muitas
características do Dhamma, como condição transcendente, e por
consequência, pode actuar como uma presença simbólica para tal, e ser
uma forma de recordar essa Verdade Última.

Por exemplo, o som \emph{nada} está sempre ``aqui''. Desta maneira, é um
bom símbolo para condição \emph{sanditthiko} do Dhamma, i.e., imanente,
presente aqui e agora''. De igual modo, não tem princípio nem fim, o que
representa muito bem a qualidade do Dhamma -- \emph{akāliko --} a
intemporalidade. É impessoal, sempre presente.

Assim que repararmos nele, instiga à investigação, ressoando, assim, o
atributo do Dhamma \emph{ehipassiko}, ``o que convida a vir ver''.

Conduz à interiorização, desmotivando o interesse de se deixar absorver
pelo mundo dos sentidos, daí a qualidade \emph{opanyiko} do Dhamma ser
igualmente bem representada.

Por fim, dá iniciativa àqueles que se interessam por presenciar e
valorizar tudo isto, daí \emph{paccattam veditabbo viññūhī --} `` a ser
conhecido por cada sábio, por si próprio''- é perfeitamente adequado a
esse atributo.

Consequentemente, ainda que seja um simples objecto sensorial, pelo
menos dentro do sistema filosófico do Budismo, os seus atributos
conferem um delicado símbolo ao Dhamma em si -- uma ressonância, se
quiser, no reino dos sentidos das qualidades fundamentais e
transcendentais da Verdade Última.

