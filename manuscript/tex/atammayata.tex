\chapter{\emph{Nada} e \emph{Atammayatā} -- Ver o Mundo Na Mente}

A terceira característica da existência, uma característica ainda mais
subtil, é chamada de \emph{`atammayatā'}. Significa literalmente ` não
feito disso'.

Quando se consideram as características do vazio e da essência, ainda
que o conceito do `eu sou' -- \emph{asmi-manā -} possa já ter sido
esclarecido, ainda podem restar alguns traços subtis de apego;
agarrarmo-nos à ideia de um mundo objectivo ser reconhecido através de
um mundo subjectivo, mesmo sabendo que não existe nenhum sentido de `eu'
discernível. Pode restar a sensação de um `isto' que conhece um `isso',
tal como dizer `sim', no caso de Realidade Intrínseca, e `não' no caso
de vazio.

\emph{Atammayatā} é o desfecho de todo esse domínio. Exprime a
realização de que, `não existe \emph{Isso}'. É o colapso genuíno tanto
da ilusão da separação entre o sujeito e o objecto, como da
discriminação entre os fenómenos, vistos como substancialmente
diferentes entre si.

Uma forma que permite desenvolver esta realização a um nível prático é a
de combinar a escuta do som \emph{nada} com a seguinte simples reflexão:

A nossa tendência é de olhar para a mente como algo que existe no corpo.
Na verdade, entendemos mal: o corpo é que existe na mente. Tudo o que
conhecemos do corpo, de agora e de antes, foi conhecido pela actividade
mental. Com isto não se pretende dizer que não existe um mundo físico,
mas o que temos como seguro, é que a experiência do corpo e a
experiência do mundo provêm da mente.

Tudo acontece aqui. E quando `o aqui' é verdadeiramente reconhecido e
despertado, a noção do mundo como algo externo, a noção de separação,
cessa. Quando nos apercebemos que o mundo está dentro de nós, a noção de
o mundo ser algo apartado de nós desvanece-se permitindo-nos uma melhor
compreensão da sua verdadeira natureza.

Se se focar no som interior e depois simplesmente reflectir, lembre-se
que `O mundo está na minha mente. O meu corpo e o mundo existem neste
espaço de consciência, permeados pelo som do silêncio', o que poderá
proporcionar-lhe uma mudança de visão. Com este domínio, acaba por ver o
seu corpo, a mente e o mundo todos numa só resolução: a compreensão da
ordenada perfeição. O mundo está equilibrado dentro desse coração pleno
de vibrante silêncio.

\emph{Atammayatā} é a premissa interna que sabe que `Não existe qualquer
``\emph{isso''}. Só existe ``\emph{isto}''.' E, quando se realiza esta
verdade, até a condição de `isto', e de `aqui' passa a não ter
significado. A presença do som \emph{nada} ajuda a realizar e a manter
tal perspectiva. Desta forma, pouco a pouco, a mente vai perdendo o
hábito de querer exteriorizar-se, de ser apanhada nas suas tendências
nefastas, \emph{āsava,} e, assim perder-se nas preocupações do mundo.
Desenvolve-se uma confortável contenção, uma compostura interna e uma
ausência de compulsões que, com tanta facilidade, perturbam o coração
confundindo e bloqueando-nos.

\emph{Atammayatā} ajuda o coração a libertar-se dos mais subtis hábitos
de inquietação e serena as reverberações das ilusões enraizadas na
dualidade sujeito-objecto. Essa tranquilidade traz ao coração a
compreensão de que só existe a integralidade do Dhamma, a noção de
espaço pleno e de realização. As aparentes dualidades disto e daquilo,
sujeito e objecto, passam a não ter qualquer significado.

