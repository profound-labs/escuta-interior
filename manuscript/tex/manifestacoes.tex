\chapter{\emph{Nada} e Suas Diferentes Manifestações}

Posto tal, é um facto que há quem tenha muita dificuldade em discernir
este som interior. Por isso, ao ler isto, poderá estar a cogitar, «Mas
afinal de que é que ele está a falar?»

Nem toda a gente consegue captar com facilidade esta experiência no
domínio da audição. Tal acontece pelos nossos traços de carácter, onde
nos deixámos condicionar de outras formas, por exemplo se for um artista
plástico, essa vibração interior pode ser mais discernível em termos de
qualidade visual, uma oscilação subtil no campo visual. Ou, caso tenha
desenvolvido uma grande consciência corporal, como um professor de hatha
yoga, poderá sentir no corpo uma delicada e penetrante condição
vibratória, um retinir, um formigar nas mãos de uma presença de energia
subtil, uma corrente contínua vital pelo corpo.

A frequência com que o captamos depende dos condicionamentos, dos
hábitos e formações provenientes de cada \emph{karma} pessoal. Da minha
experiência pessoal do ensino deste método ao longo de vinte anos, a
maioria das pessoas discerne com mais facilidade no domínio da audição.
É por isso que se chama ``\emph{nada yoga'' --} o \emph{yoga,} ou
disciplina espiritual, do som -- contudo, se tiver mais facilidade em
captar essa vibração universal pela visão, ou pelo corpo, ou até mesmo
pelo paladar ou olfacto, é uma prática com igual viabilidade. Focar-se
na sua presença e nas dinâmicas de seus efeitos, funciona exactamente da
mesma maneira, independentemente do meio sensorial através do qual se
vivencia. Pode ser usado para todas as práticas acima descritas, e ainda
assim irá obter resultados equivalentes. Se essa for a sua disposição, é
aí que vai encontrar os resultados melhores, já diz o ditado: ``O ouro
vai estar onde o encontrar''.

