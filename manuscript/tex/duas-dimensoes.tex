\chapter{As Duas Dimensões de \emph{Samādhi}}

A concentração meditativa, \emph{samādhi}, pode ser descrita como `um
objecto mental capaz de preencher a consciência durante um certo tempo',
ou ` fixar a mente num só objecto'. Por consequência, \emph{samādhi} é
uma forma de se focar num único objecto, mas que, sendo ímpar, pode
funcionar de duas maneiras. Primeiro, podemos vê-la como ` o ponto que
exclui', ou seja, reduz-se a um só objecto e põe de fora tudo o resto à
sua volta. Assim, esta forma é de uma fixação firme e estreita,
semelhante ao foco concentrado de uma lanterna ajustável. Esta é a base
de \emph{samatha}, o que significa calma ou tranquilidade.

A segunda forma pode ser chamada de `o ponto que inclui', ou seja,
trata-se de uma consciência expansiva que faz da totalidade do momento
presente o objecto de meditação. Permite-se que o `ponto único' se
expanda até englobar todos os padrões da experiência presente, tal como,
quando usamos o foco alargado da mesma lanterna ajustável todos os
objectos nesse momento são abrangidos pela luz da consciência, mais do
que existir, apenas, um foco de luz brilhante. Esta é a base do
\emph{vipassanā}, ou da realização.

Uma das grandes bênçãos da meditação do som interior é a capacidade de
suster facilmente estes dois tipos de \emph{samādhi}: tanto o ponto que
exclui, como o ponto que inclui.

