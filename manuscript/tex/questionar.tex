\chapter{\emph{Nada} e o Questionar-se}

Outra forma, quiçá mais directa, que pode ser usada ao escutar é a de
questionar-se, com o sentido de abordar e dissolver hábitos de visão
pessoal.

Mais uma vez, escute o som do silêncio, foque-se nele para centrar a
atenção com firmeza, que a mente fique o mais silenciosa e alerta
possível, e depois coloque-se a questão: «Quem sou eu?»

Inicialmente ouça o som do silêncio, a seguir questione-se e depois
aguarde; repare no que acontece quando coloca com sinceridade essa
questão «Quem sou eu?». Não estamos, explicitamente, à espera de uma
resposta verbal, de uma resposta conceptual; repare, contudo, que existe
um hiato, um hiato breve entre o tempo que decorre depois de colocarmos
a questão e antes de surgir qualquer tipo de resposta verbal,
conceptual. Quando verdadeiramente colocamos essa pergunta «Quem sou
eu?», ou «O que é que sou?», há um hiato, um espaço que se abre por
breves momentos, onde o coração intui, se abre à dúvida sobre as
presunções que temos vindo a fazer sobre ser-se alguém: homem, mulher,
novo ou velho.

Dá-se um momento de espanto antes de todos os detalhes pessoais
começarem a desaguar. Há um intervalo, uma hesitação - «Quem sou eu?»

Deixe a sua atenção repousar nesse intervalo depois do fim da pergunta e
antes de surgir a resposta. Firme a atenção nesse intervalo, nessa
dimensão, e verá, em boa verdade, que o silêncio da mente é a resposta à
questão. Permita e incentive a mente a ficar nessa amplitude aberta,
atenta e desconstruída, pois nesse momento a visão pessoal é
interrompida. Os hábitos normais de criação do ego são confusos,
reprovadores. O hábito de criar o ``eu'' é apanhado no acto. De repente
a câmara volta-se para o fotógrafo, antes que possa escapar. É o momento
desconstruído, descondicionado. A atenção surge e a mente fica alerta,
pacífica e luminosa. Mas sem qualquer sentido de eu. É algo tão
extraordinariamente simples e natural. Fixe a atenção aí.

Passado algum tempo, quando começam a surgir outras preocupações -- uma
dor na perna, o som do carro a passar, uma cócega no nariz - quando as
visões pessoais começam a reajustar-se, regresse ao som \emph{nada},
escute e ponha de novo a pergunta: «Quem sou eu?», abrindo aquela janela
da curiosidade, da realidade, perfurando a bolha da visão do ``eu'', só
por um momento. Repare no que se passa, assim que a bolha já não ofusca
ou distorce a nossa visão das coisas, e a visão pessoal cai por terra -
O que fica? O que é a vida quando se interrompe esse hábito?

Tal como a meditação no nosso nome, esta prática pode ser
simultaneamente uma ameaça e um alívio. Todavia, se não nos deixarmos
distrair por qualquer um desses sentimentos, e permanecermos
simplesmente atentos e abertos ao presente, o que se realiza é a pureza,
a radiância, a paz, uma normalidade original e uma simplicidade
abençoada, tudo envolto no silêncio ululante.

