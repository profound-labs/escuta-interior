\chapter{\emph{Nada} Como Um Sustentáculo da Realização -- \emph{Vipassanā}}

Se nos focarmos no som interior durante tempo suficiente até se tornar
firme e constante, mantendo com facilidade a mente no presente,
permitiremos então que o som se desloque lá para o fundo. Desta forma o
som é como um ecrã, onde todos os outros sons, sensações físicas,
estados de espírito e ideias são projectados -- tal como uma tela onde é
exibido o filme dos restantes padrões da nossa experiência. E por ser
plano, uniforme, é um bom ecrã. Não interfere, nem se confunde com
outros objectos que vão surgindo, e, contudo, é tão obviamente presente.
É como se fosse uma tela ligeiramente salpicada, ou uma tela especial,
na qual é projectado um filme, e por conseguinte, se estiver atento,
notará que existe um écran onde a luz é projectada. E tal vem
lembrar-lhe que só se trata de um filme - é só uma projecção, não é
real.

Podemos simplesmente deixar que o som permaneça nos bastidores, e essa
presença, por si só, é um lembrete. Sustenta a memória, ``Ah pois é: são
apenas \emph{sankhāras}, formações mentais, que chegam e partem. Todas
as formações são insatisfatórias -- \emph{sabbe sankhārā dukkhā.} Se
alguma coisa se forma, se é `algo', existe uma qualidade de
\emph{dukkha} na sua impermanência, na sua própria existência como
`algo'. Por isso não se apeguem, não se enredem, não se identifiquem,
não o vejam como vosso, ou quem são, ou o que são. Larguem.''

A presença do som pode, por conseguinte, facilitar a forma como cada
\emph{sankhāra --} seja uma sensação física, um objecto visual, um
humor, um estado refinado de felicidade, ou o que for, - seja vista como
vazio e sem dono. Tal ajuda a sustentar uma objectividade, uma
consciência liberta, uma forma de participação imparcial no presente.

Dá-se o fluir do sentir, o peso do corpo, a sensação das roupas, o fluxo
dos humores, o cansaço, a dúvida, a compreensão, a inspiração, o que
seja, mas o \emph{nada} ajuda a suster a objectividade clara por entre
os padrões de humores, de sensações e de pensamentos.

Permite que o coração repouse numa condição de consciência plena, sendo
essa mesma consciência conhecedora que recebe a corrente da experiência
-- a que sabe disto, a que se liberta dela, a que reconhece a sua
transparência, o seu vazio, a sua falta de substância.

O som interior prossegue nos bastidores, lembrando-nos que tudo é
\emph{Dhamma}, tudo é um atributo da natureza, que vem e vai, que muda,
e nada mais que isso. Esta é a verdade, que, quiçá, tenhamos intuído há
muito tempo, mas que entretanto esquecemos por nos enredarmos com a
nossa personalidade, as nossas memórias, estados de espírito e
pensamentos, desconforto do corpo e apetites.

A tensão criada pelo apego às experiências diárias desde o nascimento,
vai confundindo, iludindo e enfeitiçando a atenção. Não obstante,
podemos usar a presença do som \emph{Nada} para ajudar a quebrar a
ilusão, e terminar com o encantamento, e ajudar-nos a reconhecer a
corrente das sensações e dos humores pelo que são: padrões da natureza,
chegando e partindo, mudando, actuando à sua maneira. Não são quem, ou o
que, nós somos, e nunca nos poderão satisfazer, nem desapontar, se os
virmos intuitivamente.

