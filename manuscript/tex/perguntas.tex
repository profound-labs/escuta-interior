\chapter{Perguntas Frequentes}

\QA{P:} Não encontrei esta prática de escuta do som do silêncio mencionada
nos \emph{Suttas}, nem em qualquer comentário tradicional. De onde vem?

\smallskip
\noindent
\QA{R:} Inicialmente foi uma prática que Ajahn Sumedho pensou ter descoberto.
Ele viveu nas florestas da Tailândia durante onze anos, onde é costume
praticar meditação formal durante a noite. Essas noites estavam sempre
preenchidas com a cacofonia do barulho dos insectos, por isso,
ironicamente, só depois de ter ido viver para Londres em 1977, começou a
sentir esse som interior. Tornava-se particularmente claro no meio da
noite, no silencioso clima de neve no inverno, e um dia tornou-se
particularmente alto, até mesmo quando caminhava nas ruidosas ruas de
Haverstock Hill.

Era uma presença tão forte, que ele começou a experimentar usá-la como um
objecto de meditação, mesmo sem nunca ter ouvido falar disso antes, e, para sua
surpresa, viu nela um instrumento muito útil. Tal como escreveu no Prefácio de
uma edição recente do livro de Edward Salim Michael, ``A Lei da
Atenção''\cite{attention}:

``Descobri este `som interior' há muitos anos, mas nunca tinha ouvido ou
lido qualquer referência a tal no Canon Pali. Desenvolvi uma prática de
meditação relacionada com esta vibração em plano de fundo e obtive
grandes benefícios na expansão da consciência, conseguindo libertar
todos os pensamentos ao mesmo tempo. Alcançava uma perspectiva de
consciência transcendente que permitia reflectir nos estados mentais que
surgiam e cessavam.''

Igualmente mencionou frequentemente como o desenvolvimento deste som
interior tinha um efeito tão profundo na sua atitude para com prática da
meditação. Sendo um recém-chegado num país estrangeiro tão claramente
não-budista, e encontrando-se numa casa pequena no meio de uma grande e
ruidosa cidade, sentia muita vontade de se retirar e de escapar para a
sua amada floresta na Tailândia, de fugir para bem longe de toda aquela
multidão de `pessoas incómodas' que o perturbavam. É provável que uma
intuição vividamente clara tenha despontado no sentido de entender que
precisava de desenvolver a reclusão interna de \emph{cittaviveka}, mais
do que procurar o isolamento físico de \emph{kāyaviveka}. Além disso,
tinha descoberto esta recente prática de escuta interior, ouvindo o que
ele chamava de som do silêncio, ideal para manter esta qualidade, esta
abordagem que lhe permitia isolar-se no seu interior. Esta intuição
provou ser tão central para compreender qual a melhor maneira de
trabalhar naquele novo ambiente, que, quando tiveram de sair de Londres
para a floresta em West Sussex que tinha sido oferecida, ele chamou o
novo mosteiro de ``Cittaviveka'' -- ressoando coincidentemente com o
nome do povoado, Chithurst, onde esta nova fundação se estabeleceu.

Depois de ter usado esta prática durante alguns anos, explorando os seus
pontos altos e baixos, bem como os resultados pessoais obtidos, começou
a ensiná-la à nova comunidade do Mosteiro Cittaviveka. Ele bem sabia que
não havia qualquer referência a este método nas escrituras Theravada,
mas, uma vez que tinha tido resultados tão benéficos, sentiu que não
havia razão para não a usar.

À luz desta situação, a sua abordagem foi muito parecida com a do Ven.
Mahasi Sayadaw, que desenvolveu o seu próprio método de meditação
intuitiva nos anos 50. Alguns dos elementos do seu ``Método Mahasi''
foram criticados, dado também não pertencerem aos clássicos métodos
Theravada de meditação -- i.e., a prática da ``anotação'' escrita, e de
observar as sensações causadas pela respiração abdominal. Todavia, tal
como Ajahn Sumedho decidiu com a escuta do som do silêncio, se praticar
com estas técnicas e concluir que são uma forma de reforçar a
consciência, não será mais inteligente pô-las ao serviço, em vez de as
omitir só pela razão de não serem canónicas?

O espírito da prática budista é sempre orientado no sentido de usar
meios funcionais que possam ajudar a alcançar a libertação e o fim de
todas as insatisfações, todo o \emph{dukkha}, e, outrossim, se tal
funcionar, devemos considerá-lo como algo valioso.

\smallskip

\QA{P:} Este tipo de escuta interna já foi usado noutras tradições
espirituais? Como foi chamado de ``\emph{nada yoga''} parece, pelo
menos, ter sido descoberto por outros grupos religiosos.

\QA{R:} Após Ajahn Sumedho o ter começado a ensinar, as pessoas começaram a
mencionar-lhe como o tinham encontrado antes, quer através a sua própria
vivência, quer através de outros grupos com quem tinham meditado.
Começou então a aperceber-se que havia um conjunto muito rico de outras
tradições espirituais que o tinham usado durante séculos.

Uma das descobertas iniciais foi o livro mencionado acima, por Edward Salim
Michael. De novo, do seu Prefácio da edição de 2010: `` Lembro-me de encontrar
este livro na Sociedade Budista da Escola de Verão, há vinte e cinco anos. Tinha
a fotografia da imagem do Buda na capa e gostei do título (que antes se chamava
``A Via da Vigilância Interior''\footnote{Este livro encontra-se traduzido para
  Português pelas Publicações Maitreya}) -- e então dei uma vista de olhos ao
livro. Os capítulos do \emph{Nada yoga} intrigaram-me especialmente\ldots{}(e)
apreciei muito as instruções de E. Salim Michael sobre como integrar a plena
atenção na vida diária.''

As origens das intuições de Salim Michael no \emph{Nada yoga} foram
principalmente da sua experiência pessoal, influenciadas tanto pelas
práticas budistas, como pelo yoga hindu.

Alguns anos depois de Ajahn Sumedho ter encontrado ``A Via da Vigilância
Interior'', e ter começado a integrar alguns dos métodos contidos nessa
obra na sua prática pessoal e no ensino, contactou com outro lugar que
usava com desenvoltura esse processo.

Estava a orientar um retiro na Califórnia em 1991, num grande mosteiro
da Tradição do Budismo do Norte, chamado 'A Cidade dos Dez Mil Budas'.
Embora o retiro fosse dirigido para um grupo de 60 budistas laicos que
se tinham juntado vindos de todos os cantos dos EUA, estavam lá também
um pequeno número de monges e monjas do mosteiro anfitrião.

A meio do retiro Ajahn Sumedho introduziu a prática da escuta do som
interior. Mais ou menos um dia depois o abade recém-eleito do mosteiro,
Vem. Heng Ch'i, comentou com Ajahn Sumedho, ``Sabe, penso que veio de
encontro ao Shūrangama \emph{Samādhi.'';} Ajahn Sumedho não percebeu a
que é que se referia, pelo que o abade lhe explicou:

``Na nossa tradição Budismo Chan, a escritura-chave é o \emph{Sutra
Shūrangama} e, em particular, o ensinamento da meditação que lá se
encontra. O sutra descreve 25 exercícios espirituais que vários
Bodhisattvas apresentam ao Buda como sendo o caminho que os liberou.
Aquele que o Buda mais enaltece como sendo o mais eficaz é o do
Avalokiteshvara, Guan Shi Yin Bodhisattva. É uma meditação baseada na
audição. O Chinês do Sutra pode ser traduzido de muitas maneiras, por
isso nunca tivemos muita certeza do que significa. Guan Yin descreve o
método assim:

«Comecei com a prática baseada na iluminada natureza da audição.
Primeiro dirigi a minha audição para o interior, de forma a entrar na
corrente dos sábios\ldots{} Pelos meios descritos, entrei pelo portão da
audição e aperfeiçoei a iluminação interior do \emph{samādhi}. A minha
mente que antes tinha estado dependente de objectos da percepção,
desenvolveu perícia e desenvoltura. Por entrar na corrente dos despertos
e em \emph{samādhi}, despertei completamente. Por isto, este é o melhor
método.»\cite{surangama}

``Shifu (o Ven. Mestre Hsuan Hua, o abade fundador e professor)
explicar-nos-ia o que isto significa ao dizer-nos algo como:

«Escutar com sabedoria é escutar interiormente, não exteriormente. Não consentir
que a sua mente persiga sons. Logo de início o Buda disse para não seguirmos as
seis faculdades (olhos, ouvidos, nariz, língua, corpo e mente) e para não nos
deixarmos influenciar por elas. É preciso inverter a escuta para se ouvir a
verdadeira natureza. Em vez de dar atenção aos sons exteriores, há que focar-se
no interior quer no corpo, quer na mente, cessar a busca externa de si próprio e
virar a luz da atenção de forma a brilhar dentro de cada um.»\cite{surangama}

Nós perguntávamos o que queria dizer com `inverter a sua escuta' e ele
dizia coisas do tipo: 'Dirijam a audição para ouvirem o órgão do ouvido,
e, claro que, não é o ouvido físico -- percebem?' Mas na verdade a
maioria de nós não percebia\ldots{} Agora, com este método de escuta
interior que tem estado a ensinar nestes últimos dias, começa a tonar-se
mais claro -- especialmente frases como `a corrente dos despertos' --
compreendo agora finalmente, porque é que a prática tem sido tão
importante na nossa tradição. Obrigado por resolver o mistério do
significado dessas palavras -- obrigado por nos ensinar o modo de as
usar!''

Com o passar do tempo outras pessoas viriam a descrever práticas e
ensinamentos que usam este mesmo som interior de forma semelhante.
Recentemente a professora Zen Chozen Bays, do Mosteiro Great Vow em
Oregon, transmitiu como é que a forma dela sentir este mesmo som, e como
a sua escuta profunda são as bases do famoso \emph{koan} de Hakuin
Ekaku: `Quando duas mãos batem palmas ocorre um som; e qual é o som de
uma só mão a bater palmas? ') E ela escreveu:

``Este \emph{koan}, o Som de Uma Mão, trivializou-se no Ocidente, mas o
seu verdadeiro significado é muito profundo. O \emph{koan} é uma
pergunta que não pode ser respondida pelo vulgar método do pensamento.
Só pode ser respondida pelo não-pensamento. Pede-nos que realizemos uma
escuta profunda, ouvir como nunca o fizemos, ouvir não só com os ouvidos
mas com todo o nosso ser, olhos, pele, ossos, e coração.

A escuta profunda requer uma receptividade completa, o que significa que
nada será emitido, nada pode escapar-se. A escuta profunda pede-nos que
serenemos o corpo, a boca e a mente. Os pensamentos silenciam-se.
Impossível, dirá. Não é impossível, não é, se escutar com tanto cuidado
que até os sons dos pensamentos estão no meio da sua audição. A isto
chama-se escuta absorvente, total absorção no som.''\cite{deep}

Consequentemente, embora desconhecido para Ajahn Sumedho na altura, e
apesar de não haver referências a tal prática nos \emph{suttas} Pali,
nem nos comentários clássicos do Budismo do Sul, como no
\emph{Visudhimagga}, afinal existe na tradição budista desde há muito
tempo.

Acrescentando a todo este ancestral do mundo budista, a prática da
escuta do som interior desempenha um papel significativo em muitas
outras tradições espirituais. Por exemplo, no movimento espiritual Sant
Mat, que se originou na tradição Sikh, a `Meditação na Luz e no Som
Interior' inclui a prática de ouvir o Som Corrente, a que chamam
\emph{Shabd Naam}, ou a ` manifestação da Palavra Divina'.

Ao contrário do que se passa na tradição budista, onde o som interior
não confere grande significado espiritual por si próprio, nesta e em
muitas outras escolas é olhado como tendo uma natureza intrinsecamente
divina.

Frequentemente referido como Corrente de Vida Audível, o Som Interior,
ou o Som Corrente, o \emph{Shabd} é visto como uma essência esotérica de
Deus, disponível para todos os seres humanos -- isto de acordo com os
ensinamentos do Caminho Shabd de Eckankar, Sant Mat e Yoga Surat Shabd.

Nas palavras desta última tradição, o som interior é visto como:

``A Essência do Absoluto Ser Supremo, i.e., a força dinâmica da energia
criativa que foi enviada na aurora da manifestação do universo sob a
forma de som vibrante, pelo Ser Supremo, para o abismo do espaço, e que
tem vindo a ser enviada, ao longo das eras, enquadrando tudo quanto
constitui e habita o universo.''\cite{shabd}

O Yoga Surat Shabd descreve o seu propósito como a `União da Alma com a
Essência do Absoluto Ser Supremo'. Outras expressões para esta prática
incluem o Caminho de Luz e Som, a Jornada da Alma, e o Yoga do Som
Corrente.

O som interior tem sido desenvolvido como um caminho espiritual, ou como
ponto de referência noutras tradições. Diz-se que é usado em várias
escrituras e trabalhos filosóficos com os seguintes nomes:

\begin{itemize}
\item \emph{Naad, Akash Bani} e \emph{Sruti} nos Vedas
\item \emph{Nada} e \emph{Udgit} nas Upanishadas
\item A Música das Esferas ensinada por Pitágoras
\item \emph{Sraosha} por Zoroastro
\item \emph{Kalma} e \emph{Kalam-i-qadim} no Corão
\item \emph{Naam, Akahnd kirtan} e \emph{Sacha Shabd} no Guru Granth Sahib
\end{itemize}

\smallskip

\QA{P:} Quando sigo as suas instruções para ouvir o som do silêncio, parece
como que um tinir -- estará relacionado? Sempre vi esse som interno como
algo um pouco fastidioso, antes; agora, ao dizer-me que insista em
ouvi-lo, acabo por me deliciar com a sua presença. O que é que se está a
passar?

\QA{R:} Existe um princípio na área da construção que diz que sempre que se
confronte com uma anomalia em algum projecto onde esteja a trabalhar
(por ex: uma viga que predomina numa sala de uma casa antiga, ou uma
rocha não removível no meio de um jardim), ` Se não se puder escondê-lo,
destaque-o.'

Se viermos de uma tradição que não olha para o som \emph{nada} como uma
qualidade a exaltar e, que em vez disso, temo-nos vindo a relacionar
como sendo algo intruso e desagradável, sugiro que passemos a mudar a
nossa atitude para uma forma semelhante à do aforismo dos construtores.
E, tal como descreveu, isto corresponde ao que constatou como
verdadeiro. O que era desagradável pode tornar-se numa presença
apreciada.

Para a vasta maioria das pessoas, o som \emph{Nada} não precisa de ser
um som irritante ou intrusivo. Na verdade, tal como sentiu, num espaço
de dias ou de horas, usando uma pequena mudança de atitude, essa
incómoda rocha que estava a arruinar o seu relvado, pode ser
transformada numa feliz e encantadora presença.

Num retiro de um dia, onde estive a ensinar este tema, uma senhora disse
ao grupo que, uma vez que já conseguia ver o som \emph{Nada} como um
suporte espiritual, e sentir a sua companhia como uma verdadeira bênção,
passou a sentir uma raiva por todo o dinheiro gasto em especialistas por
causa desse som, para nada. «Que raiva que sinto!» dizia ela a rir,
«mas, estou tão aliviada por deixar de olhar para isto como um problema,
que penso que irei ultrapassá-la.»

Sobre isto, Chozen Bays escreveu:

«Muitas pessoas vêem ter comigo queixando-se que, quando meditam, são
perturbadas por um tinir ou zunir elevado nos ouvidos. Andam aflitas,
porque os médicos lhes têm dito que sofrem de uma doença incurável, o
zumbido. Quando as questiono, acabo por achar que não é zumbido, mas que
começaram a ouvir o som, chamado no Budismo Theravada, de som
\emph{Nada.} Outros têm-lhe chamado o som do verde, o som de todos os
seres vivos, ou o som entre sons. Alguns compositores afirmaram que
``A'' é o tom fundamental, e que, quando o vocalizamos estamos em
perfeita ressonância como o som essencial de toda a existência.»\cite{deep}

Para uma pequena porção de pessoas, geralmente por qualquer razão
fisiológica, o som interior é tão alto, que se torna opressivo ou
catastrófico. Nestes casos, este tipo de prática, a escuta interior, é
improvável que seja útil para a meditação, uma vez que a intensidade
subjectiva do som se torna inútil como um meio de estímulo à paz e à
clareza. À semelhança, se tiver um enfisema, com dolorosos e traiçoeiros
problemas de respiração, usar a consciência da respiração para praticar
é improvável que seja um recurso muito útil para si.

\smallskip

\QA{P:} Afinal de que é que se trata? O que causa este som? Umas tradições
vêm-no como uma presença divina, mas um fisiologista pode dizer que se
trata de um simples efeito eléctrico, disparado por impulsos neuronais
dentro dos ouvidos. O que é isto?

\QA{R:} Relativamente à prática que tenho estado a descrever, na verdade nada
disso interessa.

Um diz, ''É a Essência do Supremo Ser Absoluto'', outro, ``É só o seu
sistema nervoso a zunir.''; Pitágoras afirmaria, `` Um vez que o sol, a
lua e os planetas produzem todos a seu som particular baseado nas suas
rotações, ouvimos esta Música das Esferas, inaudível ao ouvido humano
exteriormente,'' e um praticante de Hatha Yoga, `` Não, isto é o ressoar
da energia vital, o \emph{prāna,} consoante é processado ao longo dos
sete \emph{chakras}. É a presença audível sentida pelo sistema
energético psico-fisiológico.'' `` É o Cântico da Verdade `` Não,
é\ldots{}'' E poderíamos continuar\ldots{}

O principal é não teorizarmos, usando juízos afincados que não levam a
nada, mas antes usarmos as qualidades benéficas desta vibração
omnipresente e universal que nos auxilia a despertar, bem como a ser
sábios e pacíficos.

Passa-se o mesmo com a respiração. Pode vê-la simplesmente sob a forma
científica ocidental -- os pulmões retiram a energia necessária do
oxigénio da atmosfera e expelem o desperdício do anidrido carbónico -,
ou então, olhar para a respiração como uma cósmica qualidade metafísica
-- o \emph{prāna} (que é a palavra sânscrita para `respiração') do
Universo, movendo-se em ciclos inexoráveis. Independentemente de como
classifica o seu significado -- cosmológico ou mecânico -- pode observar
a respiração e usá-la para o ajudar a concentrar-se e estar consciente.

O \emph{Nada Yoga} é comparável e essa é a atitude que eu favoreço
sempre. Independentemente do que `efectivamente' é (se conseguirmos dar
um uso significativo aqui) podemos usá-lo, e os resultados desse uso são
reais e bem tangíveis.

\smallskip

\QA{P:} Ouvi dizer que, se conseguir ouvir esse som, significa que está
iluminado -- é verdade?

Um amigo meu foi a um caríssimo fim-de-semana de meditação onde aprendeu
este método. Pode ter-lhe feito algo de bom, mas pareceu-me exagerado
chamar-lhe iluminado. O que lhe parece?

\QA{R:} Penso que este ensinamento não tem preço, mas não vale 5.000\$ por um
fim-de-semana! Pelo menos esse foi a tabela que vi para um acontecimento
desses nos EUA, há uns anos.

Definitivamente, caso ouça o som \emph{nada}, não se fica iluminado,
pelo menos da forma que a palavra é usada nos círculos budistas. Ser
iluminado, usando as definições clássicas budistas, significa que o seu
coração e a sua mente estão irreversivelmente livres de cobiça, do ódio
e da ilusão, bem como incapazes de qualquer atitude egoísta de qualquer
tipo. Os seres iluminados são seres de coração puro -- nunca agirão
dominados pelo engano, violência, desonestidade ou indulgência sensual.
Vivem num estado de inabalável paz, alegria e independência. E é bem
improvável que cobrem esse tipo de preço por seus ensinamentos.

O som \emph{nada} é uma condição de experiência natural que pode ser
vivenciada e, se for usada com sabedoria durante um extenso período de
tempo, pode tornar-se numa estratégia para alcançar a verdadeira
libertação.

É natural que as pessoas, ao terem uma experiência muito boa, se
entusiasmem a partilhá-la com os demais. Podem, por isso, exagerar as
suas manifestações através de um erro de percepção.

Da mesma forma, tendo obtido muitos benefícios, por vezes, as pessoas
querem descobrir um significado mais profundo para a sua experiência, ou
obter certa validação do exterior. Relativamente a isto, quando as
pessoas de tradição Theravada são introduzidas nesta prática de escuta
interior, por vezes exclamam coisas do tipo: ``Sabem, os discípulos do
Buda eram chamados o Sangha \emph{Sāvaka} -- `A Comunidade dos
Ouvintes'! Deve querer dizer que eles eram os únicos que conseguiam
ouvir o som interior!'' Ou até mesmo usando uma derivação etimológica
ainda mais dúbia,'' Sabem que a palavra \emph{Sotāpanna} (que significa
o que alcançou o primeiro estádio da iluminação) é sempre traduzido por
`o que entrou na corrente', mas, a meu ver, enganaram-se na tradução de
\emph{sota} -- que, sim, significa corrente, mas também significa
`orelha, o órgão da audição', usando precisamente a mesma palavra! Então
o que verdadeiramente quer dizer é:'O que realizou o Dhamma usando o
meio da audição.' Daí que, quem seja capaz de ouvir o som \emph{nada},
seja um \emph{Sotāpanna}!''

De novo, errado! Este tipo de pensamento vem do desejo uma vez mais,
pois este \emph{sota} particular, aqui, não significa isso, tal como é
corroborado em muitos outros ensinamentos.\cite{ensinamentos}

Além disso, mesmo que alguém queira contestar esta interpretação da
`entrada pelo ouvido', teria de fazer muito mais do que só ouvir o som
do silêncio para justificar chamar à experiência um sinal de ter
alcançado com ela o primeiro estado de iluminação. Para se ser um
\emph{Sotāpanna} significa que o coração e a mente estão totalmente
livres de qualquer identificação com o corpo e com a personalidade, nem
têm qualquer apego, ou confusão relativas a convenções sociais ou
religiosas, e, por fim, quem tenha alcançado esta realização ultrapassou
completamente qualquer dúvida em relação ao que é o caminho da
libertação, ou não é. Este profundo despertar está bem longe de
quaisquer bênçãos espirituais que sejam concedidas do simples facto de
ter ouvido o som \emph{nada}.

Do mesmo estilo são as afirmações que pessoas fazem como, ``É o som do
Incondicionado'', ou `` É o Cântico dos Imortais'', e até eu próprio já
fiz algumas afirmações deste tipo:

Um cântico de Verdade claro e luminoso

A ilimitada paz interior da luz

Cuja contínua presença troa

Oceânica em suas margens\cite{portrait}

O erro está em assumirmos que, por ouvirmos o som do silêncio,
`encontrámos o Imortal'. Mas não. Tudo o que podemos ter a certeza é de
que encontrámos um zunir nos ouvidos. Mais uma vez, embora estas possam
ser frases poéticas e inspiradas, e indicadores válidos na direcção da
Verdade, sob a perspectiva budista, esta presunção adicional de
consecução é uma séria sobrevalorização do caso.

No fim de tudo, considerem aqueles que pensaram que era a doença do
zumbido e que o têm tratado como uma aflição incómoda -- esses estarão
bem longe de qualquer realização espiritual, e mais ainda, também não
terão de pagar uma soma avultada para o ouvir.

Como já foi mencionado, pelo menos numa perspectiva budista, ouvir o som
do silêncio poderá ser uma estratégia para \emph{ajudar} a alcançar a
libertação, mas ouvir isso não \emph{constitui} a libertação.

O som \emph{nada} tem atributos que o tornam um som ideal do Dhamma
transcendente, o Imortal, mas é crucial manter em mente que estas
qualidades são símbolos no domínio dos sentidos para tudo quanto existe
intrinsecamente para lá dos sentidos. Se tivermos isso presente, tal
ajudar-nos-á a apreciar a sua presença, sem nos deixarmos enganar
confundindo a útil sinalética com a chegada ao fim da jornada.

\smallskip

\QA{P:} Se não conseguirmos ouvi-lo, senti-lo, vê-lo ou algo
mais\ldots{}afinal o que se faz?

\QA{R:} Bom, tal como se diz: `O ouro está onde se encontra', se não
conseguir discernir esta vibração de maneira nenhuma, poderá ser que
tenha de cavar noutro lado. Quero dizer com isto: use o método de
meditação que é mais ajustado às suas características, tal como a
concentração na respiração, ou meditação em \emph{mettā}, ou o uso de um
\emph{mantra}.

Contudo, antes de desistir há alguns meios simples que poderá tentar que
poderão ajudá-lo a o encontrar, e a seguir desenvolva a forma de os
usar. Primeiro, tente simplesmente pôr os dedos nos ouvidos. Poder-lhe-á
parecer um tanto tosco, e naturalmente não é o que gostaria de usar numa
prática de longo curso, mas pode ser um bom meio para estabelecer esse
contacto inicial -- excluir todos os sons exteriores tão completamente
quanto possível, e ver de seguida o que continua a ouvir.

Em segundo lugar, e isto é um pouco mais complexo, quando tomar o
próximo banho, ou for a uma piscina, ponha os ouvidos dentro de água e
fique quieto. De novo, dirija a sua atenção para a faculdade auditiva e
escute apenas. Não é preciso dizer que, se estiver numa piscina pública
ruidosa, é pouco provável que vá notar a diferença, mas se conseguir
fazê-lo no silêncio, poderá ser uma esclarecedora introdução ao som
interior.

