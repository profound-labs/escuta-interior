\chapter{\emph{Nada} Abrange Actividade e Compromisso}

Quando já tiver desenvolvido uma atenção estável ao som \emph{nada} na
posição formal sentada, poderá alargá-la para também fazer parte da
meditação a andar. Notará que, embora de olhos abertos e com o corpo a
caminhar firmemente para a frente e para trás entre os dois limites do
trilho definido para a meditação a andar, consegue ouvir o som
\emph{nada} abrangendo tudo. Lá está ele, solidamente lá no fundo,
permeando toda a experiência e relembrando-o que tudo isto se torna
entendível dentro da esfera da sua consciência. O corpo e o mundo estão
indubitavelmente dentro da mente.

Ao tornar-se um adepto na manutenção da atenção usando o som do silêncio
nestes variados objectos, entenderá que poderá usá-lo em quase todas as
situações. A sua atenção torna-se mais robusta.

Se estiver a caminhar na rua, a brincar com os filhos, à espera de uma
reunião de negócios, a comer uma refeição, em pé numa fila, a falar com
amigos, a ver TV, a escrever um artigo, ao visitar a sua mãe\ldots{} até
mesmo no meio de uma actividade ruidosa ou na presença de barulhos
estridentes, como trânsito intenso, a proximidade de uma serra eléctrica
ou de um martelo pneumático, se escutar, lá está ele. Podemos, assim,
usá-lo sempre como um suporte para a plena atenção e perfeita
consciência.

Além disso, se for usado como um lembrete para obter perspectivas mais
adequadas, ajuda a relacionar-se com a actividade em questão, de uma
forma mais sensível. Parece que amplia a atenção, mais do que dividi-la.
Acrescido a isto, ao prestar-lhe atenção no meio das actividades e dos
compromissos, permite-lhe viver as situações sem a obstrução das
preocupações egóicas.

Está a conceder a si próprio uma oportunidade de responder
conscientemente aos inumeráveis acontecimentos e experiências da vida,
de acordo com as leis da natureza, mais do que a reagir cegamente, por
via dos hábitos e das compulsões. Pode libertar-se dos infindáveis
ciclos de impulsos e de arrependimentos, nos quais a maioria de nós se
sente enredado.

