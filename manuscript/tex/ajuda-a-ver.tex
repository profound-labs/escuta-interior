\chapter{\emph{Nada} Ajuda a Ver Pela Visão Pessoal}

Uma das obstruções principais a tão ilimitadas atitudes, é a fiel
perturbadora, visão pessoal. Felizmente podemos usar o som interior, o
\emph{nada}, para reforçar a visão desse hábito mental criador de
pessoalidade, bem como a obsessão para o regenerar continuamente.

Uma prática que pode ajudar o coração a libertar-se de tal impulso, é
meditar no seu próprio nome. Comece por escutar o som interior por um
momento. Concentre-se nisso até a mente aclarar e se abrir, vazia, e
então pronuncie simplesmente o seu nome, internamente, qualquer que seja
o nome. Antes escutara o som do silêncio, depois o som do silêncio
dentro, e depois por detrás, do seu nome, e por fim o som do silêncio
após o ter repetido. 'A-ma-ro', `Su-san', `John'. Veja o que é que esse
som lhe traz. É só o som do seu nome, tão familiar, tão vulgar; para
variar, veja o que acontece quando ele se deixa cair no silêncio da
mente e é realmente sentido e percebido. Veja o que é que faz, se
descortina o hábito de se ver sob alguma forma em particular, abrindo
fronteiras. Para nossa surpresa, esse nome, essas sílabas tão
familiares, subitamente pode sentir-se como a mais peculiar, a mais
estranha formulação do mundo. Algo se agita e intui no coração, «Afinal
que relação é que isso tem com algo real?». Nesse momento compreendemos
que a palavra que forma o nosso nome e que é normalmente usado para nos
referir é, efectivamente, uma condição completamente impessoal.
Proferir, desta maneira, o nosso nome, no inequívoco espaço aberto da
sabedoria, pode ser percebido como tentar escrevê-lo com um feixe de luz
numa catarata. Não existe nada com que registar, nem onde registar.

Este tipo de prática pode ser tão ligeiramente perturbador, quão
gloriosamente libertador, e se consentirmos que nos liberte, tudo o que
resta é esse sabor de liberdade, e o som da água a cair.

