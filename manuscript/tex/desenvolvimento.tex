\chapter{\emph{Nada} e o Desenvolvimento Da Compaixão}

Além de ajudar o coração na libertação de tendências tão obstrutivas, e
de defender qualidades tão saudáveis no meio de actividades e
compromissos, a presença do som \emph{nada} também pode ser usada para
estimular e manter a bondade e a compaixão. Se considerarmos o quanto
recebemos e nos empenhamos com o mundo em geral, estes são os predicados
mais abençoados e proveitosos que devemos cultivar.

É de realçar que o Bodhisattava Guan Yin, Avalokiteshvara, na tradição
do Budismo do Norte, corresponde ao papel da encarnação da compaixão. O
seu nome significa `Aquele que Escuta os Sons do Mundo', e tendo em
conta isto, dá-nos uma notável indicação sobre a origem das raízes da
compaixão. Ainda que possamos registar o conceito de compaixão como
sendo prioritariamente `actuar no sentido de ajudar os seres que
sofrem', este nome (e, sem dúvida, a prática de meditação recomendada
por Guan Yin, como descrita abaixo) aponta para o predicado central,
como sendo mais de receptividade e de harmonização com o estado das
coisas. Assim, através de tamanha e atenciosa aceitação, as mil mãos de
Guan Yin podem dar início ao trabalho.

As características do Bodhisattva são um símbolo espiritual orientador
de caminhos possíveis de trabalhar. Podemos assumir a prática de ouvir o
som interior e usá-lo como forma de ajudar a expressar compaixão na
vida. Abrindo o coração ao som do silêncio e libertando-nos de outras
preocupações, conseguimos estar em plena consciência e sabiamente
atentos ao momento presente e a tudo o que ele contém; usando essa plena
consciência, a disposição inata de compaixão, no coração puro, desperta:
e então, essa atitude compassiva toca os seres que nos rodeiam.
Acrescido a isto, o simples treino de escutar tem o seu próprio impacto
na forma como nos relacionamos com os outros. Foi já explicado como o
ouvir o som do silêncio ajuda na observação dos pensamentos; ora bem,
tal acaba por ser igualmente eficaz na capacidade de escutar os outros.
A bondade e a compaixão requerem muita paciência e aceitação, e a
prática de saber escutar é um meio poderoso para as fazer germinar e
moldar. Tentar ouvir os outros -- sem reagir, sem se envolver, sem se
desligar, sem se enfadar -- é uma arte e uma graça. Acompanhar o que o
outro está a dizer e, nisto, recebê-lo completamente, é uma bênção para
ele e para si.

Numa visão alargada, podemos estender esta atitude de atenção
compassiva, e passar a escutar os sons do mundo, de tal forma que o
coração aprenda a abraçar todos os seres e suas labutas. Note-se que não
se trata de um abraço hipotético, mas antes -- tal como o
Avalokiteshvara não só ouve, mas possui muitas cabeças, mãos, olhos e
competências -- de harmonizar os nossos corações com o mundo inteiro
originando actos e palavras que auxiliem sob formas mais práticas e
tangíveis. Ao aprender a escutar o som do silêncio desta maneira -- sem
paixão, aversão ou enfado -- estamos a incrementar uma via directa para
a bondade e a compaixão, atitudes que concedem um sublime espaço eterno
ao coração, e que iluminam o mundo com tanta beleza.

