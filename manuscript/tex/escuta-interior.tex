\chapter{Escuta Interior}

Há um certo número de temas que são familiares a quem pratica a
meditação budista: `consciência da respiração', focando-se no ritmo da
respiração; `meditação a andar' que gira à volta dos passos que se dá
para cima e para baixo; a repetição interna de um mantra, como por
exemplo, `Bud-dho' -- todos eles servem para fixar a atenção na presença
do próprio momento, na realidade presente.

A par destes métodos mais conhecidos existem muitos outros com uma
função semelhante. Um deles é conhecido como `escuta interior' ou
`meditação no som do silêncio', ou em sânscrito: \emph{nada yoga}. Toda
esta terminologia refere-se a estar atento ao que se chamou o 'som do
silêncio', ou o `som \emph{nada}'.

\emph{Nada} é a palavra sânscrita que significa `som', mas também a palavra
espanhola e portuguesa para `a ausência de algo' -- uma interessante e acidental
coincidência plena de significado.

% TODO review: a palavra espanhola e portuguesa -- note not necessary
% *N.T.(nota da tradutora)

\section{O Som Interior e Como Encontrá-lo}

O som \emph{nada} é um tom interior de vibração de elevada frequência.
Se dirigirem a atenção para a audição, se ouvirem com muito cuidado os
sons à sua volta, escutarão um som contínuo de alta frequência, como que
um som branco -- sem começo, sem fim -- cintilando bem dentro, lá no
fundo.

Vejam se conseguem discernir esse som e focar-se nele. De momento não é
preciso teorizar sobre ele ou divagar sobre o que será exactamente,
decidam apenas centrar-se nele. Tentem detectar essa tão delicada
vibração.

Se forem capazes de ouvir esse som interior, poderão usá-lo tão
simplesmente como qualquer uma outra forma de meditação. Pode ser usado
tal como a respiração como um objecto de consciencialização. Centrem
nele a atenção e permitam que preencha toda a esfera da vossa
consciência.

\section{As Duas Dimensões de Samādhi}

A concentração meditativa, \emph{samādhi}, pode ser descrita como `um
objecto mental capaz de preencher a consciência durante um certo tempo',
ou ` fixar a mente num só objecto'. Por consequência, \emph{samādhi} é
uma forma de se focar num único objecto, mas que, sendo ímpar, pode
funcionar de duas maneiras. Primeiro, podemos vê-la como ` o ponto que
exclui', ou seja, reduz-se a um só objecto e põe de fora tudo o resto à
sua volta. Assim, esta forma é de uma fixação firme e estreita,
semelhante ao foco concentrado de uma lanterna ajustável. Esta é a base
de \emph{samatha}, o que significa calma ou tranquilidade.

A segunda forma pode ser chamada de `o ponto que inclui', ou seja,
trata-se de uma consciência expansiva que faz da totalidade do momento
presente o objecto de meditação. Permite-se que o `ponto único' se
expanda até englobar todos os padrões da experiência presente, tal como,
quando usamos o foco alargado da mesma lanterna ajustável todos os
objectos nesse momento são abrangidos pela luz da consciência, mais do
que existir, apenas, um foco de luz brilhante. Esta é a base do
\emph{vipassanā}, ou da realização.

Uma das grandes bênçãos da meditação do som interior é a capacidade de
suster facilmente estes dois tipos de \emph{samādhi}: tanto o ponto que
exclui, como o ponto que inclui.

\section{Nada Como Sustentáculo da Tranquilidade -- Samatha}

Podemos fazer do som interior um objecto fundamental de atenção,
permitindo que preencha todo o espaço do conhecido. Com plena
consciência, largamos tudo o mais -- as sensações no corpo, os ruídos
que ouvimos, os pensamentos que surgem -- colocando-os na periferia, nas
orlas do nosso foco de interesse. Em sua substituição, permitimos que o
som interior preencha completamente o foco da nossa atenção, o espaço
desta consciência, sustentando directamente a consolidação de
\emph{samatha}, a tranquilidade. Podemos usá-lo tal como fazemos com a
respiração, para dominar a atenção, e ser um objecto único que ajuda a
centrar, estabilizar, a dar firmeza - uma unidade de atenção no
presente.

\section{Nada Como Um Sustentáculo da Realização -- Vipassanā}

Se nos focarmos no som interior durante tempo suficiente até se tornar
firme e constante, mantendo com facilidade a mente no presente,
permitiremos então que o som se desloque lá para o fundo. Desta forma o
som é como um ecrã, onde todos os outros sons, sensações físicas,
estados de espírito e ideias são projectados -- tal como uma tela onde é
exibido o filme dos restantes padrões da nossa experiência. E por ser
plano, uniforme, é um bom ecrã. Não interfere, nem se confunde com
outros objectos que vão surgindo, e, contudo, é tão obviamente presente.
É como se fosse uma tela ligeiramente salpicada, ou uma tela especial,
na qual é projectado um filme, e por conseguinte, se estiver atento,
notará que existe um écran onde a luz é projectada. E tal vem
lembrar-lhe que só se trata de um filme - é só uma projecção, não é
real.

Podemos simplesmente deixar que o som permaneça nos bastidores, e essa
presença, por si só, é um lembrete. Sustenta a memória, ``Ah pois é: são
apenas \emph{sankhāras}, formações mentais, que chegam e partem. Todas
as formações são insatisfatórias -- \emph{sabbe sankhārā dukkhā.} Se
alguma coisa se forma, se é `algo', existe uma qualidade de
\emph{dukkha} na sua impermanência, na sua própria existência como
`algo'. Por isso não se apeguem, não se enredem, não se identifiquem,
não o vejam como vosso, ou quem são, ou o que são. Larguem.''

A presença do som pode, por conseguinte, facilitar a forma como cada
\emph{sankhāra --} seja uma sensação física, um objecto visual, um
humor, um estado refinado de felicidade, ou o que for, - seja vista como
vazio e sem dono. Tal ajuda a sustentar uma objectividade, uma
consciência liberta, uma forma de participação imparcial no presente.

Dá-se o fluir do sentir, o peso do corpo, a sensação das roupas, o fluxo
dos humores, o cansaço, a dúvida, a compreensão, a inspiração, o que
seja, mas o \emph{nada} ajuda a suster a objectividade clara por entre
os padrões de humores, de sensações e de pensamentos.

Permite que o coração repouse numa condição de consciência plena, sendo
essa mesma consciência conhecedora que recebe a corrente da experiência
-- a que sabe disto, a que se liberta dela, a que reconhece a sua
transparência, o seu vazio, a sua falta de substância.

O som interior prossegue nos bastidores, lembrando-nos que tudo é
\emph{Dhamma}, tudo é um atributo da natureza, que vem e vai, que muda,
e nada mais que isso. Esta é a verdade, que, quiçá, tenhamos intuído há
muito tempo, mas que entretanto esquecemos por nos enredarmos com a
nossa personalidade, as nossas memórias, estados de espírito e
pensamentos, desconforto do corpo e apetites.

A tensão criada pelo apego às experiências diárias desde o nascimento,
vai confundindo, iludindo e enfeitiçando a atenção. Não obstante,
podemos usar a presença do som \emph{Nada} para ajudar a quebrar a
ilusão, e terminar com o encantamento, e ajudar-nos a reconhecer a
corrente das sensações e dos humores pelo que são: padrões da natureza,
chegando e partindo, mudando, actuando à sua maneira. Não são quem, ou o
que, nós somos, e nunca nos poderão satisfazer, nem desapontar, se os
virmos intuitivamente.

\section{Nada Permite Escutar Os Pensamentos}

Consoante vai desenvolvendo a escuta interior, usando a meditação
formal, começa a notar como o facto de escutar um objecto audível ajuda
a escutar de forma objectiva os pensamentos, os estados de espírito.

De certa forma, o tagarelar do pensamento não tem mais significado que o
sibilar cintilante do som \emph{nada}. São tão-somente as vibrações da
mente pensante dando forma a padrões conceptuais, só isso -- apenas uma
murmurante corrente de vibrações prolongada e contínua. Podemos, assim,
aprender a ouvir o nosso pensamento, tal como escutaríamos uma corrente
de água, uma queda de água, ou uma música coral de um bando de aves, com
o mesmo tipo de ausência de envolvimento ou de identificação. Não é mais
que um ribeiro murmurante na mente. Nada de mais - não suscita nada, nem
perturba.

Tudo isto é muito fácil de dizer, mas efectivamente temos a tendência
para nos envolver nas nossas histórias, não é? Adoramos histórias,
particularmente histórias sobre nós: o bem e o mal que fizemos, o
memorável, o doloroso, o lamentável, o que queremos fazer, o que
esperamos, o que receamos, o que os outros pensam de nós\ldots{}

Estes prezados padrões são todos manifestações do elemento 'eu', hábitos
de pensamento de toda uma vida, em termos de `eu', `mim 'e `meu'. Em
Pali são chamados de \emph{ahamkāra}, `criador de eu' e
\emph{mamankāra}, 'criador de meu', sendo eles as características-chave
da visão pessoal. Estes hábitos são o que mais repetida e efectivamente
dirige a nossa atenção para o domínio conceptual, levando, por sua vez,
à dispersão da mente. Se a história tiver um eu, torna-se muito mais
interessante que qualquer outra história remota. Tudo isto é
extremamente natural, um hábito básico de todos nós.

Por conseguinte, grande parte da meditação intuitiva, o desenvolvimento
de \emph{vipassanā,} só trata disso - aprender a reconhecer os hábitos
dos `criadores do eu e do meu' nos pensamentos que temos.
\emph{Ahamkāra} significa literalmente `feito de si próprio', enquanto
\emph{mamankāra} significa `feito de mim próprio'; a verdadeira intuição
é o reconhecimento desses hábitos, não se deixar levar pela história --
saber ver neles o vazio e a transparência, e libertar-se deles.

\section{Nada, Vazio e Verdade}

A maioria dos praticantes do Budismo, independentemente da tradição,
estão familiarizados como que se chama `as três características da
existência' -- \emph{anicca, dukkha, anattā} (impermanência,
insatisfação e não-eu). Estas são as qualidades universais de todas as
experiências que surgem e desaparecem, e admitir a sua presença é o
aspecto mais vivo da meditação \emph{vipassanā.}

Existem, contudo, outras características universais da existência que, à
semelhança, podem ser usadas para ajudar o coração a libertar-se de toda
a limitação, peso e tensão. Duas destas características, operando como
num par, são chamadas de \emph{suññatā} e \emph{tathatā}, significando
respectivamente, vazio e Verdade. O termo `vazio' deriva de dizer `não'
ao fenómeno do mundo: «Não vou acreditar nisto -- é oco, vazio, não tem
nada, não é propriamente real.»

A `Verdade' é uma qualidade que corresponde a `vazio', da mesma forma
que a direita combina com a esquerda. Contrastando, porém, com a sua
parceira a sua natureza deriva de dizer `sim' ao universo. Poderá não
haver aqui nada de sólido, separado ou individual -- quer seja um
pensamento, um narciso ou uma montanha -- existe, contudo, algo, há aqui
uma Realidade Última que sustenta, permeia, abrange e constitui tudo. A
palavra `Verdade' expressa, assim, um apreço à natureza dessa Realidade,
e a sua tomada de consciência pode ser caracterizada pelo conhecimento e
materialização da presença do Incondicionado, o Imortal, ou
\emph{amata-dhamma}.

Quando se fala de vazio no \emph{Canon Pali,} as escrituras do mundo do
Budismo do sul, geralmente significa `o vazio do eu e do que pertence ao
eu', mas também se refere à insubstancialidade dos objectos. Quando se
desenvolve a capacidade da escuta interior e de seguir o som
\emph{nada}, potencia-se a compreensão quer do sujeito, quer do objecto,
quer do `eu', quer do `outro'.

Quando se firma na escuta do som \emph{nada} de uma forma estável, sendo
o seu tom cintilante de prata já uma presença constante, percebe-se quão
fácil é o reconhecimento da ausência de substância de todo o `eu',
`mim', ou 'meu', origem das atitudes e dos pensamentos, como já descrito
anteriormente. É como que uma luz brilhante, através da qual conseguimos
ver com clareza o oco nas bolhas de ar que flutuam.

De forma análoga, para todos os objectos mentais que são vividos -- tais
como o que vemos, ouvimos, cheiramos, saboreamos e tocamos, bem como as
memórias, planos, humores e ideias que surgem na mente -- a presença do
som \emph{nada} ajuda a iluminar a transparência de todos estes padrões
de consciência. O Buda disse-o desta forma:

A forma material é como espuma

Tocando uma bolha de água;

A percepção é só uma miragem,

Volições semelhantes a um tronco de planta,

Consciência, um truque de magia --

Assim diz o Parente do Sol.

Contudo, deve-se ponderar

Ou indagar cuidadosamente,

Afinal tudo é vazio e vago

Quando visto verdadeiramente

\emph{\textasciitilde{}S 22.95}

O som \emph{nada} também pode ajudar a relembrar a Verdade de todas as
experiências. Embora estes predicados possam parecer contraditórios,
será mais correcto dizer que são complementares. Quando se escuta
atentamente o som do silêncio e se permite que preencha o espaço
interior da consciência, a sua qualidade energética juntamente com a
riqueza informe de sua presença, é uma forte lembrança intuitiva da
condição da Verdade. É como se (pelo menos para quem fala inglês) o som
interior se expressasse num `\emph{iiiissssss}'\ldots{}, ou
\emph{`thussss\ldots{}}' infinito, que o traz de novo à realidade.

A Verdade é, por definição, conceptualmente difícil de explicar, possui
uma característica intrinsecamente incompreensível que pode parecer vaga
ou irreal, mas que, ironicamente, se torna parte necessária do seu
significado. É bem significativo o facto de Buda ter atribuído a si
próprio a expressão \emph{Tathāgata} -- que significa tanto ` O que
alcançou a `Verdade' como ` O que foi para a Verdade', dependendo da
interpretação. Assim, mesmo que a palavra `Verdade' possa trazer um tom
intangível (ao som do silêncio), tal é deliberado, e deve ser
reconhecida como expressão de uma realidade fundamental.

Pode-se fazer uma comparação com o mundo da matemática, usando o
conceito da raiz quadrada de -1. No mundo dos números reais não há
nenhum número inteiro que se possa multiplicar por si próprio para
produzir -1. Se, contudo, tal número existisse, todo o tipo de
possibilidades interessantes se abririam, como foi descoberto há muito
tempo, e desenvolvido pelos matemáticos do séc.XVIII.

É intrigante, mesmo sabendo que este número não existe no mundo real, só
tendo um estatuto imaginário, como se torna essencial na construção dos
deslocamentos de fase (atraso de propagação) dos osciladores, usado na
engenharia do som, estendendo-se o seu uso aos gráficos informatizados,
à robótica, ao processamento de sinal, simulações informatizadas e à
mecânica orbital.

Em conclusão, tal como acontece com a essência, mesmo que seja
indescritível, tem uma presença clara e demonstrável no mundo real. (1)

\section{Nada e `Atammayatā' -- Ver o Mundo Na Mente}

A terceira característica da existência, uma característica ainda mais
subtil, é chamada de \emph{`atammayatā'}. Significa literalmente ` não
feito disso'.

Quando se consideram as características do vazio e da essência, ainda
que o conceito do `eu sou' -- \emph{asmi-manā -} possa já ter sido
esclarecido, ainda podem restar alguns traços subtis de apego;
agarrarmo-nos à ideia de um mundo objectivo ser reconhecido através de
um mundo subjectivo, mesmo sabendo que não existe nenhum sentido de `eu'
discernível. Pode restar a sensação de um `isto' que conhece um `isso',
tal como dizer `sim', no caso de Realidade Intrínseca, e `não' no caso
de vazio.

\emph{Atammayatā} é o desfecho de todo esse domínio. Exprime a
realização de que, `não existe \emph{Isso}'. É o colapso genuíno tanto
da ilusão da separação entre o sujeito e o objecto, como da
discriminação entre os fenómenos, vistos como substancialmente
diferentes entre si.

Uma forma que permite desenvolver esta realização a um nível prático é a
de combinar a escuta do som \emph{nada} com a seguinte simples reflexão:

A nossa tendência é de olhar para a mente como algo que existe no corpo.
Na verdade, entendemos mal: o corpo é que existe na mente. Tudo o que
conhecemos do corpo, de agora e de antes, foi conhecido pela actividade
mental. Com isto não se pretende dizer que não existe um mundo físico,
mas o que temos como seguro, é que a experiência do corpo e a
experiência do mundo provêm da mente.

Tudo acontece aqui. E quando `o aqui' é verdadeiramente reconhecido e
despertado, a noção do mundo como algo externo, a noção de separação,
cessa. Quando nos apercebemos que o mundo está dentro de nós, a noção de
o mundo ser algo apartado de nós desvanece-se permitindo-nos uma melhor
compreensão da sua verdadeira natureza.

Se se focar no som interior e depois simplesmente reflectir, lembre-se
que `O mundo está na minha mente. O meu corpo e o mundo existem neste
espaço de consciência, permeados pelo som do silêncio', o que poderá
proporcionar-lhe uma mudança de visão. Com este domínio, acaba por ver o
seu corpo, a mente e o mundo todos numa só resolução: a compreensão da
ordenada perfeição. O mundo está equilibrado dentro desse coração pleno
de vibrante silêncio.

\emph{Atammayatā} é a premissa interna que sabe que `Não existe qualquer
``\emph{isso''}. Só existe ``\emph{isto}''.' E, quando se realiza esta
verdade, até a condição de `isto', e de `aqui' passa a não ter
significado. A presença do som \emph{nada} ajuda a realizar e a manter
tal perspectiva. Desta forma, pouco a pouco, a mente vai perdendo o
hábito de querer exteriorizar-se, de ser apanhada nas suas tendências
nefastas, \emph{āsava,} e, assim perder-se nas preocupações do mundo.
Desenvolve-se uma confortável contenção, uma compostura interna e uma
ausência de compulsões que, com tanta facilidade, perturbam o coração
confundindo e bloqueando-nos.

\emph{Atammayatā} ajuda o coração a libertar-se dos mais subtis hábitos
de inquietação e serena as reverberações das ilusões enraizadas na
dualidade sujeito-objecto. Essa tranquilidade traz ao coração a
compreensão de que só existe a integralidade do Dhamma, a noção de
espaço pleno e de realização. As aparentes dualidades disto e daquilo,
sujeito e objecto, passam a não ter qualquer significado.

\section{Nada Abrange Actividade e Compromisso}

Quando já tiver desenvolvido uma atenção estável ao som \emph{nada} na
posição formal sentada, poderá alargá-la para também fazer parte da
meditação a andar. Notará que, embora de olhos abertos e com o corpo a
caminhar firmemente para a frente e para trás entre os dois limites do
trilho definido para a meditação a andar, consegue ouvir o som
\emph{nada} abrangendo tudo. Lá está ele, solidamente lá no fundo,
permeando toda a experiência e relembrando-o que tudo isto se torna
entendível dentro da esfera da sua consciência. O corpo e o mundo estão
indubitavelmente dentro da mente.

Ao tornar-se um adepto na manutenção da atenção usando o som do silêncio
nestes variados objectos, entenderá que poderá usá-lo em quase todas as
situações. A sua atenção torna-se mais robusta.

Se estiver a caminhar na rua, a brincar com os filhos, à espera de uma
reunião de negócios, a comer uma refeição, em pé numa fila, a falar com
amigos, a ver TV, a escrever um artigo, ao visitar a sua mãe\ldots{} até
mesmo no meio de uma actividade ruidosa ou na presença de barulhos
estridentes, como trânsito intenso, a proximidade de uma serra eléctrica
ou de um martelo pneumático, se escutar, lá está ele. Podemos, assim,
usá-lo sempre como um suporte para a plena atenção e perfeita
consciência.

Além disso, se for usado como um lembrete para obter perspectivas mais
adequadas, ajuda a relacionar-se com a actividade em questão, de uma
forma mais sensível. Parece que amplia a atenção, mais do que dividi-la.
Acrescido a isto, ao prestar-lhe atenção no meio das actividades e dos
compromissos, permite-lhe viver as situações sem a obstrução das
preocupações egóicas.

Está a conceder a si próprio uma oportunidade de responder
conscientemente aos inumeráveis acontecimentos e experiências da vida,
de acordo com as leis da natureza, mais do que a reagir cegamente, por
via dos hábitos e das compulsões. Pode libertar-se dos infindáveis
ciclos de impulsos e de arrependimentos, nos quais a maioria de nós se
sente enredado.

\section{Nada e o Desenvolvimento Da Compaixão}

Além de ajudar o coração na libertação de tendências tão obstrutivas, e
de defender qualidades tão saudáveis no meio de actividades e
compromissos, a presença do som \emph{nada} também pode ser usada para
estimular e manter a bondade e a compaixão. Se considerarmos o quanto
recebemos e nos empenhamos com o mundo em geral, estes são os predicados
mais abençoados e proveitosos que devemos cultivar.

É de realçar que o Bodhisattava Guan Yin, Avalokiteshvara, na tradição
do Budismo do Norte, corresponde ao papel da encarnação da compaixão. O
seu nome significa `Aquele que Escuta os Sons do Mundo', e tendo em
conta isto, dá-nos uma notável indicação sobre a origem das raízes da
compaixão. Ainda que possamos registar o conceito de compaixão como
sendo prioritariamente `actuar no sentido de ajudar os seres que
sofrem', este nome (e, sem dúvida, a prática de meditação recomendada
por Guan Yin, como descrita abaixo) aponta para o predicado central,
como sendo mais de receptividade e de harmonização com o estado das
coisas. Assim, através de tamanha e atenciosa aceitação, as mil mãos de
Guan Yin podem dar início ao trabalho.

As características do Bodhisattva são um símbolo espiritual orientador
de caminhos possíveis de trabalhar. Podemos assumir a prática de ouvir o
som interior e usá-lo como forma de ajudar a expressar compaixão na
vida. Abrindo o coração ao som do silêncio e libertando-nos de outras
preocupações, conseguimos estar em plena consciência e sabiamente
atentos ao momento presente e a tudo o que ele contém; usando essa plena
consciência, a disposição inata de compaixão, no coração puro, desperta:
e então, essa atitude compassiva toca os seres que nos rodeiam.
Acrescido a isto, o simples treino de escutar tem o seu próprio impacto
na forma como nos relacionamos com os outros. Foi já explicado como o
ouvir o som do silêncio ajuda na observação dos pensamentos; ora bem,
tal acaba por ser igualmente eficaz na capacidade de escutar os outros.
A bondade e a compaixão requerem muita paciência e aceitação, e a
prática de saber escutar é um meio poderoso para as fazer germinar e
moldar. Tentar ouvir os outros -- sem reagir, sem se envolver, sem se
desligar, sem se enfadar -- é uma arte e uma graça. Acompanhar o que o
outro está a dizer e, nisto, recebê-lo completamente, é uma bênção para
ele e para si.

Numa visão alargada, podemos estender esta atitude de atenção
compassiva, e passar a escutar os sons do mundo, de tal forma que o
coração aprenda a abraçar todos os seres e suas labutas. Note-se que não
se trata de um abraço hipotético, mas antes -- tal como o
Avalokiteshvara não só ouve, mas possui muitas cabeças, mãos, olhos e
competências -- de harmonizar os nossos corações com o mundo inteiro
originando actos e palavras que auxiliem sob formas mais práticas e
tangíveis. Ao aprender a escutar o som do silêncio desta maneira -- sem
paixão, aversão ou enfado -- estamos a incrementar uma via directa para
a bondade e a compaixão, atitudes que concedem um sublime espaço eterno
ao coração, e que iluminam o mundo com tanta beleza.

\section{Nada Ajuda a Ver Pela Visão Pessoal}

Uma das obstruções principais a tão ilimitadas atitudes, é a fiel
perturbadora, visão pessoal. Felizmente podemos usar o som interior, o
\emph{nada}, para reforçar a visão desse hábito mental criador de
pessoalidade, bem como a obsessão para o regenerar continuamente.

Uma prática que pode ajudar o coração a libertar-se de tal impulso, é
meditar no seu próprio nome. Comece por escutar o som interior por um
momento. Concentre-se nisso até a mente aclarar e se abrir, vazia, e
então pronuncie simplesmente o seu nome, internamente, qualquer que seja
o nome. Antes escutara o som do silêncio, depois o som do silêncio
dentro, e depois por detrás, do seu nome, e por fim o som do silêncio
após o ter repetido. 'A-ma-ro', `Su-san', `John'. Veja o que é que esse
som lhe traz. É só o som do seu nome, tão familiar, tão vulgar; para
variar, veja o que acontece quando ele se deixa cair no silêncio da
mente e é realmente sentido e percebido. Veja o que é que faz, se
descortina o hábito de se ver sob alguma forma em particular, abrindo
fronteiras. Para nossa surpresa, esse nome, essas sílabas tão
familiares, subitamente pode sentir-se como a mais peculiar, a mais
estranha formulação do mundo. Algo se agita e intui no coração, «Afinal
que relação é que isso tem com algo real?». Nesse momento compreendemos
que a palavra que forma o nosso nome e que é normalmente usado para nos
referir é, efectivamente, uma condição completamente impessoal.
Proferir, desta maneira, o nosso nome, no inequívoco espaço aberto da
sabedoria, pode ser percebido como tentar escrevê-lo com um feixe de luz
numa catarata. Não existe nada com que registar, nem onde registar.

Este tipo de prática pode ser tão ligeiramente perturbador, quão
gloriosamente libertador, e se consentirmos que nos liberte, tudo o que
resta é esse sabor de liberdade, e o som da água a cair.

\section{Nada e o Questionar-se}

Outra forma, quiçá mais directa, que pode ser usada ao escutar é a de
questionar-se, com o sentido de abordar e dissolver hábitos de visão
pessoal.

Mais uma vez, escute o som do silêncio, foque-se nele para centrar a
atenção com firmeza, que a mente fique o mais silenciosa e alerta
possível, e depois coloque-se a questão: «Quem sou eu?»

Inicialmente ouça o som do silêncio, a seguir questione-se e depois
aguarde; repare no que acontece quando coloca com sinceridade essa
questão «Quem sou eu?». Não estamos, explicitamente, à espera de uma
resposta verbal, de uma resposta conceptual; repare, contudo, que existe
um hiato, um hiato breve entre o tempo que decorre depois de colocarmos
a questão e antes de surgir qualquer tipo de resposta verbal,
conceptual. Quando verdadeiramente colocamos essa pergunta «Quem sou
eu?», ou «O que é que sou?», há um hiato, um espaço que se abre por
breves momentos, onde o coração intui, se abre à dúvida sobre as
presunções que temos vindo a fazer sobre ser-se alguém: homem, mulher,
novo ou velho.

Dá-se um momento de espanto antes de todos os detalhes pessoais
começarem a desaguar. Há um intervalo, uma hesitação - «Quem sou eu?»

Deixe a sua atenção repousar nesse intervalo depois do fim da pergunta e
antes de surgir a resposta. Firme a atenção nesse intervalo, nessa
dimensão, e verá, em boa verdade, que o silêncio da mente é a resposta à
questão. Permita e incentive a mente a ficar nessa amplitude aberta,
atenta e desconstruída, pois nesse momento a visão pessoal é
interrompida. Os hábitos normais de criação do ego são confusos,
reprovadores. O hábito de criar o ``eu'' é apanhado no acto. De repente
a câmara volta-se para o fotógrafo, antes que possa escapar. É o momento
desconstruído, descondicionado. A atenção surge e a mente fica alerta,
pacífica e luminosa. Mas sem qualquer sentido de eu. É algo tão
extraordinariamente simples e natural. Fixe a atenção aí.

Passado algum tempo, quando começam a surgir outras preocupações -- uma
dor na perna, o som do carro a passar, uma cócega no nariz - quando as
visões pessoais começam a reajustar-se, regresse ao som \emph{nada},
escute e ponha de novo a pergunta: «Quem sou eu?», abrindo aquela janela
da curiosidade, da realidade, perfurando a bolha da visão do ``eu'', só
por um momento. Repare no que se passa, assim que a bolha já não ofusca
ou distorce a nossa visão das coisas, e a visão pessoal cai por terra -
O que fica? O que é a vida quando se interrompe esse hábito?

Tal como a meditação no nosso nome, esta prática pode ser
simultaneamente uma ameaça e um alívio. Todavia, se não nos deixarmos
distrair por qualquer um desses sentimentos, e permanecermos
simplesmente atentos e abertos ao presente, o que se realiza é a pureza,
a radiância, a paz, uma normalidade original e uma simplicidade
abençoada, tudo envolto no silêncio ululante.

\section{Os Atributos De Nada}

Vários atributos do som \emph{nada} manifestam qualidades espirituais
muito úteis, algumas das quais permitem que sejam, pelo menos,
universalmente acessíveis e profícuas no que se refere, quanto mais não
seja, à concentração na respiração.

Primeiro, ao usar o som \emph{nada} como um objecto de meditação,
estimula-se a atitude de escuta e receptividade. Exige que, mais do que
dirigir uma actividade, se vivencie de coração aberto.

Segundo, o som não está sujeito ao controlo pessoal. Diferente da
respiração, que podemos alongar, encurtar, ou mudar segundo a nossa
vontade, não podemos tornar o som interior mais estridente ou mais
suave, fazer com que acabe ou comece, ou qualquer outra coisa. Podemos
focar-nos nele ou não, mas não está sujeito a direcções ou escolhas
pessoais. Na verdade, incita-nos naturalmente à realização na mais
profunda impessoalidade -- sem qualquer característica particular que
nos leve a pensar em ``mim'' ou ``meu''. Não é feminino, nem masculino,
novo nem velho, inteligente nem estúpido\ldots{}não tem tamanho nem
nacionalidade, nem cor nem língua\ldots{} existe, tão simplesmente, na
imparcialidade da Natureza.

Por fim, é energizante, possui uma qualidade que estimula naturalmente.
Quanto mais atenções lhe dermos, mais lúcida se torna a mente. Funciona
como um circuito de feedback positivo, de tal forma que, quanto maior
atenção se lhe der, mais reforçada será a capacidade de ficar atento.
Suporta, assim, o próprio acto de meditar, ao ajudar a mente a ficar
mais alerta.

\section{Nada Como Um Símbolo De Transcendência}

O som do silêncio é um objecto do domínio sensorial que reflecte muitas
características do Dhamma, como condição transcendente, e por
consequência, pode actuar como uma presença simbólica para tal, e ser
uma forma de recordar essa Verdade Última.

Por exemplo, o som \emph{nada} está sempre ``aqui''. Desta maneira, é um
bom símbolo para condição \emph{sanditthiko} do Dhamma, i.e., imanente,
presente aqui e agora''. De igual modo, não tem princípio nem fim, o que
representa muito bem a qualidade do Dhamma -- \emph{akāliko --} a
intemporalidade. É impessoal, sempre presente.

Assim que repararmos nele, instiga à investigação, ressoando, assim, o
atributo do Dhamma \emph{ehipassiko}, ``o que convida a vir ver''.

Conduz à interiorização, desmotivando o interesse de se deixar absorver
pelo mundo dos sentidos, daí a qualidade \emph{opanayiko} do Dhamma ser
igualmente bem representada.

Por fim, dá iniciativa àqueles que se interessam por presenciar e
valorizar tudo isto, daí \emph{paccattam veditabbo viññūhī --} `` a ser
conhecido por cada sábio, por si próprio''- é perfeitamente adequado a
esse atributo.

Consequentemente, ainda que seja um simples objecto sensorial, pelo
menos dentro do sistema filosófico do Budismo, os seus atributos
conferem um delicado símbolo ao Dhamma em si -- uma ressonância, se
quiser, no reino dos sentidos das qualidades fundamentais e
transcendentais da Verdade Última.

\section{Nada e Suas Diferentes Manifestações}

Posto tal, é um facto que há quem tenha muita dificuldade em discernir
este som interior. Por isso, ao ler isto, poderá estar a cogitar, «Mas
afinal de que é que ele está a falar?»

Nem toda a gente consegue captar com facilidade esta experiência no
domínio da audição. Tal acontece pelos nossos traços de carácter, onde
nos deixámos condicionar de outras formas, por exemplo se for um artista
plástico, essa vibração interior pode ser mais discernível em termos de
qualidade visual, uma oscilação subtil no campo visual. Ou, caso tenha
desenvolvido uma grande consciência corporal, como um professor de hatha
yoga, poderá sentir no corpo uma delicada e penetrante condição
vibratória, um retinir, um formigar nas mãos de uma presença de energia
subtil, uma corrente contínua vital pelo corpo.

A frequência com que o captamos depende dos condicionamentos, dos
hábitos e formações provenientes de cada \emph{karma} pessoal. Da minha
experiência pessoal do ensino deste método ao longo de vinte anos, a
maioria das pessoas discerne com mais facilidade no domínio da audição.
É por isso que se chama ``\emph{nada yoga'' --} o \emph{yoga,} ou
disciplina espiritual, do som -- contudo, se tiver mais facilidade em
captar essa vibração universal pela visão, ou pelo corpo, ou até mesmo
pelo paladar ou olfacto, é uma prática com igual viabilidade. Focar-se
na sua presença e nas dinâmicas de seus efeitos, funciona exactamente da
mesma maneira, independentemente do meio sensorial através do qual se
vivencia. Pode ser usado para todas as práticas acima descritas, e ainda
assim irá obter resultados equivalentes. Se essa for a sua disposição, é
aí que vai encontrar os resultados melhores, já diz o ditado: ``O ouro
vai estar onde o encontrar''.
