\chapter{Escuta Interior}

Há um certo número de temas que são familiares a quem pratica a
meditação budista: `consciência da respiração', focando-se no ritmo da
respiração; `meditação a andar' que gira à volta dos passos que se dá
para cima e para baixo; a repetição interna de um mantra, como por
exemplo, `Bud-dho' -- todos eles servem para fixar a atenção na presença
do próprio momento, na realidade presente.

A par destes métodos mais conhecidos existem muitos outros com uma
função semelhante. Um deles é conhecido como `escuta interior' ou
`meditação no som do silêncio', ou em sânscrito: \emph{nada yoga.} Toda
esta terminologia refere-se a estar atento ao que se chamou o 'som do
silêncio', ou o `som \emph{nada'.}

\emph{Nada} é a palavra sânscrita que significa `som', mas também a
palavra espanhola (* e portuguesa) para `a ausência de algo' -- uma
interessante e acidental coincidência plena de significado.

*N.T.(nota da tradutora)
