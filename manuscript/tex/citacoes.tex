\chapter{Citações}

\setlength{\parindent}{0pt}
\setlength{\parskip}{0.5\baselineskip}

\section*{O Shūrangama Sūtra}

{\small\itshape

  Excerto do \emph{Sūrangama Sūtra} recentemente traduzido do Chinês pelo Comité
  de Traduções do Sūrangama Sūtra da Sociedade de Traduções de Textos Budistas.
}

\smallskip

``Comecei por uma prática baseada nessa luminosa natureza que é a
audição. Primeiro dirigi o meu ouvido internamente de forma a entrar na
corrente dos sábios. A seguir os sons exteriores desapareceram. Uma vez
a direcção da minha escuta invertida e os sons silenciados, quer os
sons, quer o silêncio, deixaram de existir. Gradualmente, conforme
progredia, era isso que ouvia, e a consciência do que ouvia chegou ao
fim. E até mesmo quando esse estado de espírito, em que tudo tinha
acabado, desaparecia, eu não descansava. A minha consciência e os
objectos da minha consciência estavam vazios, e quando esse processo de
esvaziamento da minha consciência estava completo, então até esse
esvaziar e tudo quanto tinha sido esvaziado, desapareciam. O vir à
existência, e o cessar de existir como sendo eles próprios, cessava. E
aí, a quietude última, revelava-se.

``De repente ao transcender os mundos dos seres comuns, também
transcendi o mundo dos seres que transcenderam os mundos vulgares. Tudo,
nas dez direcções, se iluminou completamente, e eu alcancei dois poderes
notáveis. Primeiro, a minha mente ascendeu para se unir à fundamental, à
maravilhosa e iluminada mente de todos os Budas nas dez direcções, e o
meu poder de compaixão tornou-se igual ao deles. Depois, a minha mente
desceu para se unir a todos os seres dos seis destinos em todas as dez
direcções, de tal forma, que senti seus sofrimentos e pedidos em orações
como sendo meus.

\clearpage

\begin{quotation}
``Agora, respeitosamente afirmo ao Honrado pelo Mundo\\
Aquele que se tornou Buda neste mundo \emph{Sahā}\\
Para nos transmitir a verdade, ensinamento essencial,\\
Neste lugar -- afirmo que a pureza se revela pela audição.\\
Todos aqueles que desejam alcançar a maestria do \emph{samādhi}\\
Certamente concordarão que a audição é o meio de entrada.''

\emph{(p. 253)}
\end{quotation}

\begin{quotation}
``Grande Assembleia! Ānanda! Cesse o espectáculo de marionetes\\
Que é vossa escuta distorcida! Dirijam a vossa audição apenas\\
Para ouvir a vossa verdadeira e genuína natureza,\\
Que é o destino da Via Suprema,\\
Este é o caminho genuíno que vencerá obstáculos até à iluminação.''

\emph{(p. 256)}
\end{quotation}

\clearpage

\section*{A Upanishad Chandogya}

``A luz que brilha acima deste céu, por detrás de todos, por detrás de
tudo, nos mais altos mundos onde não há outros mais altos -- na verdade
essa é a mesma luz que existe aqui no interior de cada pessoa. Existe
nesta escuta -- quando fechamos os olhos e se ouve um som, um bramido,
semelhante a um crepitar do fogo.''

\emph{(Up Ch 3.13.78)}

{\small\itshape
  Citação em \emph{Mind Like Fire Unbound} / \emph{A Mente como Fogo Sem Limite}
  Ch.1, tradução de Ven. Thanissaro Bhikku.
}

\section*{Os Itinerantes do Dharma}

O silêncio é tão intenso que se pode ouvir o ribombar do nosso sangue
nos ouvidos, mas bastante mais alto que isso é o misterioso bramir que
sempre identifiquei como o troar do diamante da sabedoria, o misterioso
troar do próprio silêncio, que é um imenso \emph{'shhh'} recordando-nos
de algo que parece termos esquecido na agitação diária desde que
nascemos.

Como eu gostaria de o explicar àqueles que amo, à minha mãe, a Japhy,
mas nunca encontrei palavras para descrever esse nada, essa pureza.
«Haverá algum ensinamento definido para ser dado a todos os seres
vivos?» foi a pergunta que eventualmente foi colocada ao Dipankara com
sobrancelhas castanhas já embranquecidas, e a sua resposta foi: o
troante silêncio do diamante.

\emph{The Dharma Bums} Cap.22, por Jack Kerouac

\clearpage

\section*{A Subida ao Monte Carmelo}

Nada, nada, nada, nada, nada

E até na montanha, nada.

\emph{A Subida ao Monte Carmelo -- A Via do Espírito Puro}, S. João da
Cruz

\setlength{\parskip}{0pt}
\setlength{\parindent}{17pt}
